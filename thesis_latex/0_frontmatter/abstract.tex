
% Thesis Abstract -----------------------------------------------------
% Keywords command
\providecommand{\keywords}[1]
{
	\small	
	\textbf{\textit{Keywords---}} #1
}

%\begin{abstractslong}    %uncommenting this line, gives a different abstract heading
\begin{abstracts}        %this creates the heading for the abstract 
	Later a long time has seen the development of machine learning (ML) methods for business enhancement over different areas. Organizations are in need to improve their business process performance by utilizing predictive models for ongoing business cases. Predictive process monitoring (PPM) tackles this problem by forecasting the behaviour, execution, and outcome of business processes at runtime. PPM approaches take an event log (i.e. a collection of complete business cases) as input and utilize ML methods to teach predictive models on how to predict the future status of a specific business process (incomplete trace). Future state to be predicted of the ongoing case can vary based on the problem objective, e.g. Will the loan application will be approved or declined (i.e. final outcome)? What is the next event given previous events? Or What is the remaining time until the end?. In specific, a family of approaches of PPM known as outcome-oriented PPM that focuses on anticipating whether a business case will end with an expected outcome or not. An outcome-oriented PPM framework is expected to form precise predictions in the early execution stages to decide if the system
	worker should take part and get involved or not to avoid unexpected outcomes. Given what has been said, this thesis addresses the question of how to improve the predictive process monitoring of business process outcomes. To answer this question, we propose three different enhancements to the currently existing approaches that have been introduced in the literature. Additionally, this work reports on a comparative exploratory assessment of existing methods, utilizing a benchmark covering 20 prediction tasks that comes from different real-life event logs. Empirical results confirm that our proposed techniques achieved significant improvement to the current existing PPM techniques in terms of accuracy.
	
	
\keywords{Predictive process monitoring, Process Mining,  Machine Learning}

\textbf{CERCS:} P170 -  Computer science, numerical analysis, systems, control \\

{  \bfseries \Large {Pealkiri: }\large Täiustatud klassifikaatori koolitusmeetodid ennustatava protsessi jälgimiseks}\\[0.5cm]

{\Large \bfseries L\"{u}hikokkuv\~{o}te:}

Hiljem on pikka aega arenenud masina\~{o}ppe meetodid (ML) ettev\~{o}tluse edendamiseks erinevates valdkondades. Organisatsioonid peavad parandama oma \"{a}riprotsesside toimivust, kasutades ennustatavaid mudeleid käimasolevate \"{a}rijuhtumite jaoks. Ennustav protsesside j\"{a}lgimine (PPM) aitab selle probleemiga toime tulla, prognoosides \"{a}riprotsesside k\"{a}itumist, t\"{a}itmist ja tulemusi t\~{o}\~{o}ajas. PPM-l\"{a}henemised v\~{o}tavad sisendina sündmuste logi (s.o t\"{a}ielike \"{a}rijuhtumite kogumi) ja kasutavad ML-meetodeid, et \~{o}petada ennustavaid mudeleid konkreetse \"{a}riprotsessi tulevase oleku ennustamiseks (mittetäielik j\"{a}lg). K\"{a}imasoleva juhtumi ennustatav tulevane seisund v\~{o}ib varieeruda s\~{o}ltuvalt probleemi eesm\"{a}rgist, nt. Kas laenutaotlus v\~{o}etakse vastu v\~{o}i l\"{u}katakse tagasi (s.o l\~{o}pptulemus)? Milline on j\"{a}rgmine s\"{u}ndmus, v\~{o}ttes arvesse varasemaid s\"{u}ndmusi? V\~{o}i mis on l\~{o}puni j\"{a}\"{a}nud aeg ?. T\"{a}psemalt \~{o}eldes on PPM-i l\"{a}henemisviiside perekond, mida tuntakse kui tulemustele orienteeritud PPM-i ja mis keskendub prognoosimisele, kas \"{a}rijuhtum l\~{o}peb oodatava tulemusega v\~{o}i mitte. Eeldatakse, et tulemustele orienteeritud PPM-raamistik moodustab varajastes t\"{a}itmisetappides t\"{a}psed ennustused, et otsustada, kas s\"{u}steemit\~{o}\~{o}taja peaks sellest osa v\~{o}tma ja sellest osa saama v\~{o}i mitte, et v\"{a}ltida ootamatuid tulemusi. Arvestatut arvestades k\"{a}sitletakse k\"{a}esolevas l\~{o}putöös k\"{u}simust, kuidas parandada \"{a}riprotsesside tulemuste ennustavat protsesside j\"{a}lgimist. Sellele k\"{u}simusele vastamiseks pakume v\"{a}lja kolm erinevat t\"{a}iendust kirjanduses kasutusele v\~{o}etud olemasolevatele l\"{a}henemisviisidele. Lisaks antakse k\"{a}esolevas t\~{o}\~{o}s \"{u}levaade olemasolevate meetodite v\~{o}rdlevast uurimuslikust hinnangust, kasutades v\~{o}rdlusalust, mis h\~{o}lmab 20 ennustusülesannet, mis p\"{a}rinevad erinevatest reaalajas aset leidnud s\"{u}ndmuste logidest. Empiirilised tulemused kinnitavad, et meie v\"{a}ljapakutud tehnikad parandasid m\"{a}rkimisv\"{a}\"{a}rselt praeguste PPM-meetodite t\"{a}psust.
\\
\\
{\Large \bfseries V\~{o}tmes\~{o}nad:}

Prognoositav protsesside j\"{a}lgimine, protsesside kaevandamine, masin\"{o}pe
\\
\\
{\large \bfseries CERCS: P170}  - Arvutiteadus, arvutusmeetodid, s\"{u}steemid, juhtimine (automaatjuhtimisteooria)




\end{abstracts}
%\end{abstractlongs}

\let\cleardoublepage\clearpage
% ---------------------------------------------------------------------- 
